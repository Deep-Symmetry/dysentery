\documentclass[11pt]{article}

\usepackage[english]{babel}
\usepackage{courier}
\usepackage[bottom]{footmisc}
\usepackage{bytefield}
\usepackage{hyperref}

\title{DJ Link Packet Analysis}
\author{James Elliott\\Deep Symmetry, LLC}

\begin{document}

\maketitle

\abstract{The protocol used by Pioneer professional DJ equipment to
  communicate and coordinate performances can be monitored to provide
  useful information for synchronizing other software, such as light
  shows and sequencers. By creating a ``virtual CDJ'' that sends
  appropriate packets to the network, other devices can be induced to
  send packets containing even more useful information about their
  state. This paper documents what has been learned so far about the
  protocol, and how to accomplish these tasks.}

\section{Mixer Startup}

When the mixer starts up, after it obtains an IP address (or gives up
on doing that and self-assigns an address), it sends out what look
like a series of packets\footnote{The packet capture described in this
  analysis can be found at
  \url{https://github.com/brunchboy/dysentery/raw/master/doc/assets/powerup.pcapng}}
simply announcing its existence to UDP port 50000 on the broadcast
address of the local network.

\begin{figure}
  \begin{bytefield}[bitwidth=1.5em]{16}
    \bitheader{0-15} \\
    \begin{rightwordgroup}{Name}
      \begin{leftwordgroup}{Header}
        \bitboxes*{1}{{\tt 51} {\tt 73} {\tt 70} {\tt 74} {\tt 31} {\tt 57} {\tt 6d} {\tt 4a} {\tt 4f}
          {\tt 4c} {\tt \textbf{0a}} {\tt 00}}
        & \bitbox[lrt]{4}{}
      \end{leftwordgroup} \\
      \wordbox[lrb]{1}{Device Name (padded with {\tt 00})} 
    \end{rightwordgroup} \\
    \bitboxes*{1}{{\tt 01} {\tt 02} {\tt 00} {\tt 25} {\tt 02}} \\
  \end{bytefield}
  \caption{Initial announcement packets from Mixer}
  \label{fig:mixerInitial}
\end{figure}

These have a data length\footnote{Values within packets are shown in
  hexadecimal, while packet lengths and byte offsets are discussed in
  decimal.} of 37 bytes, appear roughly every 300 milliseconds, and
have the content shown in Figure \ref{fig:mixerInitial}.

The tenth byte (inside what is labeled the header) is bolded because
its value changes in the different types of packets which follow.

After about three of these packets are sent, another series of three
begins. It is not clear what purpose these packets serve, because they
are not yet asserting ownership of any device number; perhaps they are
used when CDJs are powering up as part of the mechanism the mixer can
use to tell them which device number to use based on which network
port they are connected to?

\begin{figure}
  \begin{bytefield}[bitwidth=1.5em]{16}
    \bitheader{0-15} \\
    \begin{rightwordgroup}{Name}
      \begin{leftwordgroup}{Header}
        \bitboxes*{1}{{\tt 51} {\tt 73} {\tt 70} {\tt 74} {\tt 31} {\tt 57} {\tt 6d} {\tt 4a} {\tt 4f}
          {\tt 4c} {\tt \textbf{00}} {\tt 00}}
        & \bitbox[lrt]{4}{}
      \end{leftwordgroup} \\
      \wordbox[lrb]{1}{Device Name (padded with {\tt 00})} 
    \end{rightwordgroup} \\
    \bitboxes*{1}{{\tt 01} {\tt 02} {\tt 00} {\tt 2c} {\emph{N}} {\tt 02}} &
    \bitbox{6}{MAC address} \\
  \end{bytefield}
  \caption{First-stage Mixer device number assignment packets}
  \label{fig:mixerStage1}
\end{figure}

In any case, these three packets have a data length of 44 bytes, are
again sent to UDP port 50000 on the local network broadcast address,
at roughly 300 millisecond intervals, and have the content shown in
Figure \ref{fig:mixerStage1}.

The value \emph{N} at byte 36 is 1, 2, or 3, depending on whether this
is the first, second, or third time the packet is sent.

After these comes another series of three numbered packets. These
appear to be claiming the device number for a particular device, as
well as announcing the IP address at which it can be found. They have
a data length of 50 bytes, and are again sent to UDP port 50000 on the
local network broadcast address, at roughly 300 millisecond intervals,
with the content shown in Figure \ref{fig:mixerStage2}.

\begin{figure}[ht]
  \begin{bytefield}[bitwidth=1.5em]{16}
    \bitheader{0-15} \\
    \begin{rightwordgroup}{Name}
      \begin{leftwordgroup}{Header}
        \bitboxes*{1}{{\tt 51} {\tt 73} {\tt 70} {\tt 74} {\tt 31} {\tt 57} {\tt 6d} {\tt 4a} {\tt 4f}
          {\tt 4c} {\tt \textbf{02}} {\tt 00}}
        & \bitbox[lrt]{4}{}
      \end{leftwordgroup} \\
      \wordbox[lrb]{1}{Device Name (padded with {\tt 00})} 
    \end{rightwordgroup} \\
    \bitboxes*{1}{{\tt 01} {\tt 02} {\tt 00} {\tt 32}} &
    \bitbox{4}{IP address} & \bitbox{6}{MAC address} &
    \bitbox{1}{\emph{D}} & \bitbox{1}{\emph{N}} \\
    \bitboxes*{1}{{\tt 02} {\tt 01}} \\
  \end{bytefield}
  \caption{Second-stage Mixer device number assignment packets}
  \label{fig:mixerStage2}
\end{figure}

I identify these as claiming/identifying the device number because the
value \emph{D} at byte 46 is the same as the device number that the
mixer uses to identify itself ({\tt 0x21}) and the same is true for the
corresponding packets seen from the CDJs (they use device numbers 2
and 3, as they are connected to those ports/channels on the mixer).

As with the previous series of three packets, the value \emph{N} at
byte 47 takes on the values 1, 2, and 3 in the three packets.

These are followed by another three packets, perhaps the last stage of
claiming the device number, again at 300 millisecond intervals, to the
same port 50000. These shorter packets have 38 bytes of data and the
content shown in Figure \ref{fig:mixerStage3}.

\begin{figure}
  \begin{bytefield}[bitwidth=1.5em]{16}
    \bitheader{0-15} \\
    \begin{rightwordgroup}{Name}
      \begin{leftwordgroup}{Header}
        \bitboxes*{1}{{\tt 51} {\tt 73} {\tt 70} {\tt 74} {\tt 31} {\tt 57} {\tt 6d} {\tt 4a} {\tt 4f}
          {\tt 4c} {\tt \textbf{04}} {\tt 00}}
        & \bitbox[lrt]{4}{}
      \end{leftwordgroup} \\
      \wordbox[lrb]{1}{Device Name (padded with {\tt 00})} 
    \end{rightwordgroup} \\
    \bitboxes*{1}{{\tt 01} {\tt 02} {\tt 00} {\tt 26}} &
    \bitbox{1}{\emph{D}} & \bitbox{1}{\emph{N}} \\
  \end{bytefield}
  \caption{Final-stage Mixer device number assignment packets}
  \label{fig:mixerStage3}
\end{figure}

As before the value \emph{D} at byte 36 is the same as the device
number that the mixer uses to identify itself ({\tt 0x21}) and
\emph{N} at byte 37 takes on the values 1, 2, and 3 in the three
packets.

Once those are sent, the mixer seems to settle down and send what
looks like a keep-alive packet to retain presence on the network and
ownership of its device number, at a less frequent interval. These
packets are 54 bytes long, again sent to port 50000 on the local
network broadcast address, roughly every second and a half. They have
the content shown in Figure \ref{fig:mixerKeepalive}.

\begin{figure}[h]
  \begin{bytefield}[bitwidth=1.5em]{16}
    \bitheader{0-15} \\
    \begin{rightwordgroup}{Name}
      \begin{leftwordgroup}{Header}
        \bitboxes*{1}{{\tt 51} {\tt 73} {\tt 70} {\tt 74} {\tt 31} {\tt 57} {\tt 6d} {\tt 4a} {\tt 4f}
          {\tt 4c} {\tt \textbf{06}} {\tt 00}}
        & \bitbox[lrt]{4}{}
      \end{leftwordgroup} \\
      \wordbox[lrb]{1}{Device Name (padded with {\tt 00})} 
    \end{rightwordgroup} \\
    \bitboxes*{1}{{\tt 01} {\tt 02} {\tt 00} {\tt 36}} &
    \bitbox{1}{\emph{D}} & \bitbox{1}{\tt 02} &
    \bitbox{6}{MAC address} & \bitbox{4}{IP address} \\
    \bitboxes*{1}{{\tt 01} {\tt 00} {\tt 00} {\tt 00} {\tt 02} {\tt 00}} \\
  \end{bytefield}
  \caption{Mixer keep-alive packets}
  \label{fig:mixerKeepalive}
\end{figure}

\section{CDJ Startup}

When a CDJ starts up the procedure and packets are nearly identical,
with groups of three packets sent at 300 millisecond intervals to port
50000 of the local network broadcast address. The only difference
between Figure \ref{fig:cdjInitial} and Figure \ref{fig:mixerInitial}
is the final byte, which is {\tt 0x01} for the CDJ, and was {\tt 0x02}
for the mixer.

\begin{figure}[h]
  \begin{bytefield}[bitwidth=1.5em]{16}
    \bitheader{0-15} \\
    \begin{rightwordgroup}{Name}
      \begin{leftwordgroup}{Header}
        \bitboxes*{1}{{\tt 51} {\tt 73} {\tt 70} {\tt 74} {\tt 31} {\tt 57} {\tt 6d} {\tt 4a} {\tt 4f}
          {\tt 4c} {\tt \textbf{0a}} {\tt 00}}
        & \bitbox[lrt]{4}{}
      \end{leftwordgroup} \\
      \wordbox[lrb]{1}{Device Name (padded with {\tt 00})} 
    \end{rightwordgroup} \\
    \bitboxes*{1}{{\tt 01} {\tt 02} {\tt 00} {\tt 25} {\tt 01}} \\
  \end{bytefield}
  \caption{Initial announcement packets from CDJ}
  \label{fig:cdjInitial}
\end{figure}

Similarly, the next series of three packets from the CDJ are nearly
identical to those from the mixer. The only difference between Figure
\ref{fig:cdjStage1} and Figure \ref{fig:mixerStage1} is byte 37
(immediately after the packet counter \emph{N}), which again is {\tt
  0x01} for the CDJ, and was {\tt 0x02} for the mixer.

\begin{figure}
  \begin{bytefield}[bitwidth=1.5em]{16}
    \bitheader{0-15} \\
    \begin{rightwordgroup}{Name}
      \begin{leftwordgroup}{Header}
        \bitboxes*{1}{{\tt 51} {\tt 73} {\tt 70} {\tt 74} {\tt 31} {\tt 57} {\tt 6d} {\tt 4a}
          {\tt 4f} {\tt 4c} {\tt \textbf{00}} {\tt 00}}
        & \bitbox[lrt]{4}{}
      \end{leftwordgroup} \\
      \wordbox[lrb]{1}{Device Name (padded with {\tt 00})} 
    \end{rightwordgroup} \\
    \bitboxes*{1}{{\tt 01} {\tt 02} {\tt 00} {\tt 2c} {\emph{N}} {\tt 01}} &
    \bitbox{6}{MAC address} \\
  \end{bytefield}
  \caption{First-stage CDJ device number assignment packets}
  \label{fig:cdjStage1}
\end{figure}

However it appears that in this capture the CDJ skips the second stage
of claiming a device number, probably because it is configured to be
automatically assigned a device number based on the port of the mixer
to which it is connected, and we cannot see a packet that the mixer
sent it assigning it that device number. Instead, it jumps right to
the end of the third and final stage, sending a single 38-byte packet
with header byte 10 set to {tt 04} (which identified the three packets
of the third stage when the mixer was starting up), with content
identical to Figure \ref{fig:mixerStage3}.

Even though the value of \emph{N} is {\tt 01}, this is the only packet
in this series that the CDJ sends. It would probably behave
differently if configured to assign its own device number (behaving
like we saw the mixer behave in claiming its device number).

The CDJ then moves to the keep-alive stage, sending out 54-byte
packets with the content shown in Figure \ref{fig:cdjKeepalive}.

\begin{figure}[h]
  \begin{bytefield}[bitwidth=1.5em]{16}
    \bitheader{0-15} \\
    \begin{rightwordgroup}{Name}
      \begin{leftwordgroup}{Header}
        \bitboxes*{1}{{\tt 51} {\tt 73} {\tt 70} {\tt 74} {\tt 31} {\tt 57} {\tt 6d} {\tt 4a} {\tt 4f}
          {\tt 4c} {\tt \textbf{06}} {\tt 00}}
        & \bitbox[lrt]{4}{}
      \end{leftwordgroup} \\
      \wordbox[lrb]{1}{Device Name (padded with {\tt 00})} 
    \end{rightwordgroup} \\
    \bitboxes*{1}{{\tt 01} {\tt 02} {\tt 00} {\tt 36}} &
    \bitbox{1}{\emph{D}} & \bitbox{1}{\tt 01} &
    \bitbox{6}{MAC address} & \bitbox{4}{IP address} \\
    \bitboxes*{1}{{\tt 01} {\tt 00} {\tt 00} {\tt 00} {\tt 01} {\tt 00}} \\
  \end{bytefield}
  \caption{CDJ keep-alive packets}
  \label{fig:cdjKeepalive}
\end{figure}

As seems to always be the case when comparing mixer and CDJ packets,
the difference between this and Figure \ref{fig:mixerKeepalive} is
that byte 37 (following the device number \emph{D}) has the value {\tt
  01} rather than {\tt 02}, and the same is true of the second-to-last
byte in each of the packets. (Byte 52 is {\tt 01} in Figure
\ref{fig:cdjKeepalive} and {\tt 02} in Figure \ref{fig:mixerKeepalive}.

\section{Tracking BPM and Beats}

For some time now,
Afterglow\footnote{\url{https://github.com/brunchboy/afterglow\#afterglow}}
has been able to synchronize its light shows with music being played
on Pioneer equipment by observing packets broadcast by the mixer to
port 50001. Until recently, however, it was not possible to tell which
player was the Master, so there was no way to determine the down beat
(the start of each measure). This section will be expanded and more
details provided as Afterglow is updated to take advantage of the
discoveries described in the next section.

Until then, here is a summary of what is currently done. A socket is
opened and bound to port 50001. Whenever a packet from the mixer is
received on this socket, if the length is 96 bytes, it is known to
contain beat and BPM information. The current BPM can be obtained as:

\[ \frac{byte[90] \times 256 + byte[91]}{100} \]

These packets are sent on each beat, and the current beat number (1,
2, 3 or 4) is sent in $byte[92]$. However, the beat number is
\emph{not} synchronized with the master player, and so it is not
useful for much. We expect to make use of the Virtual CDJ technique to
determine the actual beat number soon.

\section{Creating a Virtual CDJ}

Although some useful information can be obtained simply by watching
broadcast traffic on a network containing Pioneer gear, in order to
get important details it is necessary to cause the gear to send you
information directly. This can be done by simulating a ``Virtual
CDJ''.\footnote{Thanks are due to Diogo Santos for discovering the
  trick of creating a virtual CDJ in order to receive detailed status
  information from other devices.}

To do this, bind a UDP server socket to port 50002 on the network
interface on which you are receiving DJ-Link traffic, and start
sending keep-alive packets to port 50000 on the broadcast address as
if you were a CDJ. Follow the structure shown in Figure
\ref{fig:cdjKeepalive}, but use the actual MAC and IP addresses of the
network interface on which you are receiving DJ-Link traffic, so the
devices can see how to reach you.

You can use a value like 5 for \emph{D} (the device/player number) so
as not to conflict with any actual players you have on the network,
and any name you would like. As long as you are sending these packets
roughly every 1.5 seconds, the other players and mixers will begin
sending packets directly to the socket you have opened on port 50002.

We are just beginning to analyze all the information which can be
gleaned from these packets, but here is what we know so
far.\footnote{Examples of packets discussed in this section can be
  found in the capture at
  \url{https://github.com/brunchboy/dysentery/raw/master/doc/assets/to-virtual.pcapng}}

\subsection{Mixer Status Packets}

Packets from the mixer will have a length of 56 bytes and the content
shown in Figure \ref{fig:mixerStatus}.

\begin{figure}[h]
  \begin{bytefield}[bitwidth=1.5em]{16}
    \bitheader{0-15} \\
    \begin{rightwordgroup}{Name}
      \begin{leftwordgroup}{Header}
        \bitboxes*{1}{{\tt 51} {\tt 73} {\tt 70} {\tt 74} {\tt 31} {\tt 57} {\tt 6d} {\tt 4a} {\tt 4f}
          {\tt 4c} {\tt \textbf{29}}}
        & \bitbox[lrt]{5}{}
      \end{leftwordgroup} \\
      \bitbox[lrb]{15}{Device Name (padded with {\tt 00})} \bitbox{1}{\tt 01}
    \end{rightwordgroup} \\
    \bitbox{1}{\tt 00} & \bitbox{1}{\emph{D}} & \bitboxes*{1}{{\tt 00} {\tt 14}} &
    \bitbox{1}{\emph{D}} & \bitboxes*{1}{{\tt 00} {\tt 00} {\tt d0} {\tt 00} {\tt 10} {\tt 00}
      {\tt 00} {\tt 80} {\tt 00}} & \bitbox{2}{\emph{BPM}} \\
    \bitboxes*{1}{{\tt 00} {\tt 10} {\tt 00} {\tt 00} {\tt 00} {\tt 09} {\tt 00}} &
    \bitbox{1}{\emph{X}} & \bitbox{1}{\emph{B}}
  \end{bytefield}
  \caption{Mixer status packets}
  \label{fig:mixerStatus}
\end{figure}

Packets coming from a DJM-2000 nexus connected as the only mixer on
the network contain a value of 33 ({\tt 0x21}) for their Device Number
\emph{D} (bytes 33 and 36).

The current tempo in beats-per-minute identified by the mixer can be
obtained as:

\[ \frac{byte[46] \times 256 + byte[47]}{100} \]

This value is labeled \emph{BPM} in Figure \ref{fig:mixerStatus}.

Mixer status packets are sent on each beat, and the current beat number (1,
2, 3 or 4) is sent in $byte[55]$, labeled \emph{B}. However, the beat number is
\emph{not} synchronized with the master player, and so it is not
useful for much. The beat number should be determined, when needed,
from packets that are sent by the master player.

The value at $byte[54]$, labeled \emph{X}, has an unknown meaning. It
seems to start out with the value {\tt 0x00}, and then change when a
player starts playing to the value {\tt 0xff}, but it may well do
other things as well.

\subsection{CDJ Status Packets}

Packets from a CDJ will have a length of 212 bytes and the content
shown in Figure \ref{fig:cdjStatus}.

\begin{figure}
  \begin{bytefield}[bitwidth=1.5em]{16}
    \bitheader{0-15} \\
    \begin{rightwordgroup}{Name}
      \begin{leftwordgroup}{Header}
        \bitboxes*{1}{{\tt 51} {\tt 73} {\tt 70} {\tt 74} {\tt 31} {\tt 57} {\tt 6d} {\tt 4a} {\tt 4f}
          {\tt 4c} {\tt \textbf{0a}}}
        & \bitbox[lrt]{5}{}
      \end{leftwordgroup} \\
      \bitbox[lrb]{15}{Device Name (padded with {\tt 00})} \bitbox{1}{\tt 01}
    \end{rightwordgroup} \\
    \bitbox{1}{\tt 03} & \bitbox{1}{\emph{D}} & \bitboxes*{1}{{\tt 00} {\tt b0}} &
    \bitbox{1}{\emph{D}} & \bitboxes*{1}{{\tt 00} {\tt 01} {\tt 00} {\tt 00} {\tt 00} {\tt 00}
      {\tt 00} {\tt 00} {\tt 00} {\tt 00} {\tt 00}} \\
    \wordbox[lrt]{1}{Details omitted\dots} \\
    \skippedwords \\
    \wordbox[lrb]{1}{} \\
  \end{bytefield}
  \caption{CDJ status packets (partial)}
  \label{fig:cdjStatus}
\end{figure}

This is only a partial diagram, and will be fleshed out as we progress
with our analysis. The Device Number in \emph{D} (bytes 33 and 36) is
the Player Number as displayed on the CDJ itself. In the case of this
capture, the CDJs were assigned Player Numbers 2 and 3.

We do currently know that byte 137 is a bit field
containing some very useful state flags, detailed in Figure
\ref{fig:cdjStateFlags}.

\begin{figure}[ht]
  \begin{bytefield}[endianness=big,bitwidth=4em]{8}
    \bitheader{0-7} \\
    \bitbox{1}{\tt 1} & \bitbox{1}{\small{Play}} & \bitbox{1}{\small{Master}} & \bitbox{1}{\small{Sync}}
    \bitbox{1}{\small{On-Air}} & \bitbox{1}{\tt 1} & \bitbox{1}{\tt 0} & \bitbox{1}{\tt 0} \\
  \end{bytefield}
  \caption{CDJ state flag bits}
  \label{fig:cdjStateFlags}
\end{figure}

We have not yet seen any other values for bits 0--2 or 7, so are
unsure if they also carry meaning. If you ever find different values
for them, please let us know by filing an Issue!
\url{https://github.com/brunchboy/dysentery/issues}

\end{document}

